% !TeX spellcheck = pl_PL-Polish
\documentclass[12pt,a4paper]{article}

\usepackage[T1]{fontenc}
\usepackage[polish]{babel}
\usepackage[utf8]{inputenc}
\usepackage{lmodern}
\selectlanguage{polish}
\usepackage{graphicx}
\usepackage{xcolor}
\usepackage{pgfplots}
\usepackage{tocloft}
\usepackage{geometry}
\usepackage{hyperref}
\usepackage{svg}

\geometry{margin=4cm}
\pgfplotsset{compat=1.18}
\usetikzlibrary{patterns}

\def\projectName{Zarządzanie projektem i koordynacja zespołu developerskiego przy tworzeniu aplikacji webowej z użyciem Gita, Trello i Dockera.}
\def\authorA{Paweł Lewandowski}
\def\authorB{Michał Kuta}
\newcommand\todo[1]{\textcolor{red}{#1}}

\begin{document}
\pagenumbering{gobble}
\clearpage
	\begin{figure}[h]
		\centering
		%\includegraphics[width=0.5\linewidth]{media/polsl_logo_pion_en_rgb.png}
		\includegraphics[width=0.5\linewidth]{media/ps-logo.png}
	\end{figure}

\hspace{3cm}
	\begin{center}Dokumentacja projektowa\end{center}
	\hspace{3cm}
	\begin{center}\large\textbf{Zarządzanie Systemami Informatycznymi}\end{center}
	\begin{center}\large\textit{\projectName}\end{center}

\hspace{7cm}
	\begin{flushright}Kierunek: Informatyka
		\end{flushright}
		\begin{flushright}Członkowie zespołu:
		\par
		\textit{\authorA}
		\par
		\textit{\authorB}
	\end{flushright}
\vfill
	\begin{center}Gliwice, 2024/2025\end{center}

\newpage
\pagenumbering{arabic}
\tableofcontents
\newpage

\section{Wprowadzenie}

\subsection{Role w projekcie}
W naszym zespole odpowiedzialność została rozdzielona w następujący sposób:
\begin{itemize}
	\item \textbf{Michał Kuta} -- przygotowanie szkieletu projektu, implementacja wzorca MVC, testowanie.
	\item \textbf{Paweł Lewandowski} -- organizacja pracy, zaprojektowanie bazy danych, stworzenie dokumentacji.
\end{itemize}
Podział obowiązków pozwolił na efektywną współpracę i realizację zadań zgodnie z harmonogramem.

\subsection{Cel projektu}
Celem projektu było stworzenie aplikacji umożliwiającej zapisywanie wyników rozgrywek gier planszowych. Projekt zakłada opracowanie funkcjonalnego narzędzia, które umożliwia:
\begin{itemize}
	\item Dodawanie graczy.
	\item Dodawanie gier planszowych.
	\item Zapisywanie rozgrywek wraz z wynikami.
	\item Przeglądanie historii rozgrywek.
	\item Przeglądanie rankingu graczy.
\end{itemize}
Wszystko w wygodnym dla użytkownika interfejsie.

\pagebreak
\section{Założenia projektowe}

\subsection{Założenia techniczne i nietechniczne}
\subsubsection{Założenia techniczne}
\begin{itemize}
	\item Aplikacja webowa oparta na wzorcu projektowym MVC.
	\item Backend oraz logika aplikacji zaimplementowane w języku PHP.
	\item Aplikacja uruchamiana w kontenerze Docker.
	\item Warstwa prezentacji oparta na szablonach HTML/CSS.
	\item Komunikacja z bazą danych za pomocą warstwy modelu.
	\item Dane przechowywane w relacyjnej bazie danych MySQL.
	\item Możliwość lokalnego uruchomienia poprzez docker-compose.
\end{itemize}

\subsubsection{Założenia nietechniczne}
\begin{itemize}
	\item Intuicyjny interfejs użytkownika dostosowany do komputerów i tabletów.
	\item Możliwość rejestrowania wyników wielu graczy w różnych grach.
	\item Funkcja przeglądania historii rozgrywek i statystyk.
	\item Możliwość łatwej rozbudowy o nowe funkcjonalności.
	\item Nacisk na prostotę obsługi i niezawodność.
\end{itemize}

\subsection{Stos technologiczny}
\begin{itemize}
	\item Frontend: HTML, CSS.
	\item Backend: PHP z wykorzystaniem wzorca MVC.
	\item Baza danych: MySQL.
	\item System kontroli wersji: Git, GitHub.
	\item Zarządzanie: Trello.
	\item Konteneryzacja: Docker, Docker Compose.
\end{itemize}

\subsection{Oczekiwane rezultaty projektu}
\begin{itemize}
	\item W pełni funkcjonalna aplikacja webowa umożliwiająca rejestrowanie wyników gier planszowych.
	\item Intuicyjny interfejs użytkownika umożliwiający łatwe dodawanie i przeglądanie rozgrywek.
	\item System zapisu danych z wykorzystaniem bazy danych (MySQL).
	\item Środowisko uruchomieniowe oparte na Dockerze.
	\item Dokumentacja techniczna i użytkowa aplikacji.
\end{itemize}

\section{Realizacja projektu}
Projekt został zrealizowany w następujących etapach:

\begin{enumerate}
	\item \textbf{Sprint 1 -- Planowanie:}
	\begin{itemize}
		\item Identyfikacja funkcjonalności aplikacji (dodawanie gier, graczy, przeglądanie rekordów).
		\item Utworzenie tablicy Kanban w Trello do zarządzania zadaniami.
		\item Wybór stosu technologicznego: PHP + MySQL + Docker, z wykorzystaniem wzorca MVC.
		\item Zaplanowanie struktury katalogów i komponentów aplikacji.
		\item Opracowanie planu sprintów oraz roadmapy rozwoju.
	\end{itemize}
	\item \textbf{Sprint 2 -- Tworzenie podstaw:}
	\begin{itemize}
		\item Utworzenie środowiska kontenerowego Docker (PHP + MySQL).
		\item Implementacja szkieletu aplikacji zgodnie z wzorcem MVC.
		\item Stworzenie warstwy modelu i połączenia z bazą danych.
		\item Testowanie komunikacji z bazą MySQL za pomocą PDO.
		\item Przygotowanie roboczych widoków do testowego wyświetlania danych.
	\end{itemize}
	\item \textbf{Sprint 3 -- Główne programowanie:}
	\begin{itemize}
		\item Implementacja logiki warstwy kontrolera.
		\item Zastąpienie widoków roboczych finalnymi szablonami HTML i CSS.
		\item Połączenie warstw modelu, widoku i kontrolera w spójną całość.
		\item Wprowadzenie mechanizmów obsługi błędów.
	\end{itemize}
	\item \textbf{Sprint 4 -- Testowanie:}
	\begin{itemize}
		\item Intensywne testowanie aplikacji pod kątem poprawności działania.
		\item Poprawa wykrytych błędów i usprawnienie mechanizmów walidacji.
		\item Weryfikacja stabilności komunikacji między warstwami aplikacji.
		\item Testy środowiska Docker (np. restart kontenerów, poprawność konfiguracji).
		\item \textbf{Sprint 5 -- Faza końcowa:}
		\begin{itemize}
			\item Uzupełnienie i uporządkowanie dokumentacji technicznej projektu.
			\item Weryfikacja wykonania zadań z tablicy Kanban.
			\item Finalne poprawki w kodzie oraz przygotowanie aplikacji do prezentacji.
		\end{itemize}
	\end{itemize}
\end{enumerate}
	
\pagebreak
\section{Wykresy postępu projektu}
Poniżej przedstawiono dwa wykresy ilustrujące postęp prac nad projektem w kolejnych sprintach.

\begin{figure}[h]
	\centering
	\begin{tikzpicture}
		\begin{axis}[
			width=\linewidth*0.9,
			xlabel={Sprint},
			ylabel={Pozostałe zadania},
			grid=major, grid style={dashed, gray!70},
			ymin=0, ymax=25,
			xtick={0,1,2,3,4,5},
			xticklabels={Start,Sprint 1,Sprint 2,Sprint 3,Sprint 4,Sprint 5},
			bar width=0.20cm,
			ybar,
			enlargelimits=0.1,
			]
			\addplot[
			nodes near coords,
			fill=blue,
			draw=none
			] coordinates {
				(0,22) (1,16) (2,11) (3,7) (4,3) (5,0)
			};
		\end{axis}
	\end{tikzpicture}
	\caption{Wykres liczby pozostałych zadań w kolejnych sprintach.}
\end{figure}

\pagebreak
\begin{figure}[h]
	\centering
	\begin{tikzpicture}
		\begin{axis}[
			width=\linewidth*0.9,
			xlabel={Sprint},
			ylabel={Pozostałe zadania},
			grid=major, grid style={dashed, gray!70},
			ymin=0, ymax=28,
			xmax=5,
			xtick={0,1,2,3,4,5},
			xticklabels={Start,Sprint 1,Sprint 2,Sprint 3,Sprint 4,Sprint 5},
			bar width=0.20cm,
			enlargelimits=0.1,
			legend style={at={(0.95,0.95)}, anchor=north east},
			]
			\addplot[
			ybar,
			nodes near coords,
			fill=blue,
			draw=none
			] coordinates {
				(0,22) (1,16) (2,11)
			};
			\addplot [domain=0:3, red, dashed] {-5.500*x + 21.83};
			\addplot [
			domain=3:5,
			red,
			dashed,
			mark=*,
			mark options={fill=red},
			samples at={3,4,5}
			] {-5.500*x + 21.83};
		\end{axis}
	\end{tikzpicture}
	\caption{Wykres regresji liniowej i przewidywań na przyszłe sprinty.}
\end{figure}
Według tego wykresu powinniśmy zakończyć projekt przy sprincie 4.

\pagebreak
\begin{figure}
	\centering
	\includegraphics[width=1\linewidth]{media/baza_danych}
	\caption{o a tu baza danych to sie wrzuci tam gdzie trzeba.}
	\label{fig:bazadanych}
\end{figure}


\section{Wnioski}
\begin{itemize}
	\item \textbf{Spostrzeżenia:} Praca nad aplikacją umożliwiła pogłębienie wiedzy z zakresu wzorca MVC, konteneryzacji za pomocą Dockera oraz pracy zespołowej z wykorzystaniem narzędzi takich jak Trello i GitHub. Projekt unaocznił także znaczenie jasnego podziału obowiązków oraz dobrej dokumentacji.
	\item \textbf{Osiągnięcia:} Udało się zbudować w pełni działającą aplikację webową, umożliwiającą rejestrowanie wyników gier planszowych w przejrzysty i funkcjonalny sposób. Stworzono modułowy kod z zachowaniem zasad separacji warstw, a całość została uruchomiona w środowisku Docker, co ułatwia wdrażanie i testowanie.
	\item \textbf{Potencjał rozwoju:} Projekt może być w przyszłości rozwijany o nowe funkcjonalności, takie jak: logowanie użytkowników, historia rozgrywek z filtrowaniem, generowanie statystyk i wykresów, eksport danych do plików (CSV/PDF), integracja z kontami Google lub Facebook, czy też wersja mobilna aplikacji.
\end{itemize}

\section{Bibliografia}
\begin{enumerate}
	\item Oficjalna dokumentacja PHP -- \url{https://www.php.net/docs.php}
	\item Dokumentacja Docker -- \url{https://docs.docker.com/}
	\item Dokumentacja MySQL -- \url{https://dev.mysql.com/doc/}
	\item Artykuł: „Understanding MVC Architecture" \\ -- \url{https://www.geeksforgeeks.org/mvc-design-pattern/}
	\item Oficjalna dokumentacja PDO -- \url{https://www.php.net/manual/en/book.pdo.php}
	\item Dokumentacja Git -- \url{https://git-scm.com/doc}
	\item Trello -- \url{https://trello.com}
	\item dbdiagram.io -- \url{https://dbdiagram.io}
\end{enumerate}
	
\end{document}